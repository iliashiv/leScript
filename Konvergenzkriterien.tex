\usepackage[utf8]{inputenc}
\usepackage{amssymb}
\usepackage{mathtools}
 
\title{Konvergenzkriterien}
\newtheorem{theorem}{Satz}[section]

\begin{document}

\section{Grundlagen über Reihen}
\begin{theorem}[Monotones Konvergenzkriterium für Reihen]	

Es sei $f \in \mathbb{R}^{\mathbb{N}}$ mit $f_{n} \geq 0$ für alle n. Dann konvergiert $S^{f}$ genau dann wenn $S^{f}$ nach oben beschränkt ist.

\end{theorem}

\begin{theorem}[Cauchy-Konvergenzkriterium für Reihen]

Für eine Folge $f \in \mathbb{R}^{\mathbb{N}}$ (bzw. $f \in \mathbb{C}^{\mathbb{N}}$) sind äquivalent:
\newline
(a) Die zu  $S^{f}$ gehörende Reihe konvergiert.
\newline
(b) Für jedes $\epsilon > 0$ gibt es ein $N \in \mathbb{N}$, so dass für alle $n, m \geq N$ mit $n < m$ gilt $|\Sigma^{m}_{k = n+1} f_{k}| <  \epsilon$  
 
\end{theorem}

\section{Konvergenzkriterium von Leibniz}
\begin{theorem}
Es sei $a \in \mathbb{R}^{\mathbb{N}}$ eine monoton fallende Folge. Ferner sei
$f \in \mathbb{R}^{\mathbb{N}}$ definiert durch $f_n = (-1)^n\cdot a_n$. Dann sind die folgenden beiden Aussagen äquivalent:
\newline
(a) Die zu f gehörende Reihe konvergiert.
\newline
(b) f ist eine Nullfolge
\end{theorem}

\section{Absolute Konvergenz bei Reihen}
 \begin{theorem}[Majorantenkriterium]
 
 Es seien $(a_n)_{n \in \mathbb{N}}$ und $(b_n)_{n \in \mathbb{N}}$ reel, oder -komplexwertige Folgen. Es gelte:
\newline
$\bullet |a_n| \leq |b_n|$ für jedes $n$;
\newline
$\bullet$ die zu $b$ gehörende Reihe ist konvergent
 
\end{theorem}
 
\begin{theorem}[Minorantenkriterium]
 
Es seien $(a_n)_{n \in \mathbb{N}}$ und $(b_n)_{n \in \mathbb{N}}$ reel, oder -komplexwertige Folgen. Es gelte:
\newline
$\bullet |a_n| \geq |b_n|$ für jedes $n$;
\newline
$\bullet$ die zu $b$ gehörende Reihe ist divergent
 
\end{theorem}
\newpage
\section{Wurzel- und Quotientenkriterium}
\begin{theorem}[Quotientenkriterium]
Es sei $(a_n)_{n \in \mathbb{N}}$ aus $\mathbb{R}^{\mathbb{N}}$ (oder aus $\mathbb{C}^{\mathbb{N}}$) mit $a_n \neq 0$ für fast alle $n$. Annahme, die Folge
$(Q_n)_{n \in \mathbb{N}} := (\frac{|a_{n+1}|}{|a_n|})_{n \in \mathbb{N}}$
der absoluten Quotienten konvergiert. Es gelte:

$$
\lim_{n \to \infty}\bigg(\frac{|a_{n+1}|}{|a_n|}\bigg)_{n \in \mathbb{N}} = q \in \mathbb{R}^{+}_{0} 
$$
Dann kann über die zu a gehörende Reihe $S^a$ Folgendes gesagt werden:
\newline
(1) Ist $q < 1$, so konvergiert $S^a$ absolut.
\newline
(2) Ist q > 1, so divergiert S a .
\end{theorem}

\begin{theorem}[Wurzelkriterium]

Es sei $(a_n)_{n \in \mathbb{N}}$ eine beschränkte Folge aus $\mathbb{R}^{\mathbb{N}}$ (oder aus $\mathbb{C}^{\mathbb{N}}$).
\\
Weiterhin sei:
$$
L := \limsup (\sqrt[n]{|a_n|})
$$
der Limes superior der Wurzelfolge $(W_n)_{n \in \mathbb{N}^*}$. Dann kann über die zu $a$ gehörende Reihe $S^a$ Folgendes gesagt werden:
\\
(1) Ist $L > 1$, so konvergiert $S^a absolut$
\\
(2) Ist $L < 1$, so divergiert $S^a$.

\end{theorem}

\end{document}
